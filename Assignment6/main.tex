\documentclass[12pt, letterpaper]{article}
\usepackage{fancyhdr}
\usepackage{booktabs}
\usepackage{graphicx}
\usepackage{amsmath}
\usepackage{amssymb}
\usepackage{mathtools}
\usepackage{colortbl}
\usepackage{listings}% http://ctan.org/pkg/listings
\lstset{
  basicstyle=\ttfamily,
  mathescape
}
\newcommand{\dprime}{{\prime\prime}}

\newcommand{\ptitle}{CS 621 - Assignment 6}
\newcommand{\pauthor}{Ali Hassani}

\title{\ptitle{}}
\author{\pauthor{}}

\pagestyle{fancyplain}
\fancyhf{}
\rhead{ \fancyplain{}{\pauthor{}} }
\lhead{ \fancyplain{}{\ptitle{}} }

\newcommand{\grayrow}{\rowcolor[gray]{.8}}


\renewcommand{\thesubsection}{\thesection.\alph{subsection}}

\begin{document}

\maketitle

\section{}

Let \textsc{HamPath} be the following problem: Given an undirected graph $G$, does $G$ have a Hamilton path? 
Recall that a Hamilton path is one that visits every node exactly once. 
The \textsc{HamCycle} problem has the same input but is asking whether the graph has a Hamilton cycle.

\begin{itemize}
    \item Argue that both problems are in $NP$.
    \item Show carefully that \textsc{HamCycle} $\leq^p_m$ \textsc{HamPath}.
\end{itemize}

\subsection*{Answer}

\subsubsection*{Argue that both problems are in $NP$.}
Given any graph $G=(V, E)$ with $|V| = n$ vertices, the length of any solution to $\textsc{HamCycle}$ is exactly $n$, and the length of any solution to $\textsc{HamCycle}$ is $n-1$.
The upper bound for any of these two is $n^2$ (maximum number of possible edges).
As a result, given a witness, the solution can be verified in polynomial time with respect to input size $n$.


\subsubsection*{Show carefully that \textsc{HamCycle} $\leq^p_m$ \textsc{HamPath}.}
Given any graph $G = (V, E)$, we define $f(.)$ as a function that constructs graph $f(G) = G^\prime = (V^\prime , E^\prime)$, where:
\begin{align*}
    V^\prime &= V \cap \{ x^\prime, y, y^\prime \} & \\
    E^\prime &= E \cap  \{ \{ x^\prime, a \} \ | \ \{x, a\} \in E \} \cap \{ \{ y, x \} , \{ y^\prime , x^\prime \} \}
\end{align*}
and $x$ is the least-degree vertex in $G$.
(Technically $x$ can be an arbitrary vertex, but specifying a property helps create a well-defined $f$.)
The function $f$ takes graph $G$, cleaves vertex $x$ (clones it and its edges into a new vertex $x^\prime$), and adds two additional vertices $y$ and $y^\prime$, where there's an edge between $y$ and $x$, and another between $y^\prime$ and $x^\prime$.

Through this definition, given any graph $G \in \textsc{HamCycle}$, we know that $G \in \textsc{HamPath}$ by definition (if a graph has a Hamilton cycle, it has a Hamilton path).
Adding the three new vertices does not break that.

\noindent \textbf{Proof:} Consider only the Hamilton cycle within the original graph $G$. $x$ and two edges connecting to it must be included in the cycle.
Let's name the vertices on the other end of those two edges $p$ and $q$.
Therefore $\{ x, p \}$ and $ \{ x, q \}$ are in the the cycle.
Now when we introduce $x^\prime$ into the graph, we can simply swap one of those two edges with the corresponding $x^\prime$ one, and this breaks the cycle, and creates a path.
Adding the two new vertices $y$ and $y^\prime$ only expands that path by two vertices.
As a result, adding the three new vertices and their specified edges creates a graph with a Hamilton path, given that the original graph had a Hamilton cycle.

\noindent Therefore, $G \in \textsc{HamCycle} \Longrightarrow{} f(G) \in \textsc{HamPath}$.

Now assume that for some graph $G$, $f(G) \in \textsc{HamPath}$, but $G \notin \textsc{HamCycle}$.
Given that assumption, we know the terminal vertices.
It's the $y$ and $y^\prime$ that the function $f$ introduces.
If we remove those two, we still have the cycle, and the terminal vertices are $x$ and $x^\prime$.
If $x$ and $x^\prime$ are connected with an edge, we've constructed a Hamilton cycle.
This is the same as removing $x^\prime$. If we remove $x^\prime$ however, we end up with the original graph, $G$.
Therefore $G \in \textsc{HamCycle}$, and that is a contradiction.

\noindent As a result, $f(G) \in \textsc{HamPath} \Longrightarrow{} G \in \textsc{HamCycle}$.

Therefore, we can conclude that $G \in \textsc{HamCycle} \Longleftrightarrow{} f(G) \in \textsc{HamPath}$.

\vspace{5mm}

Therefore, \textsc{HamCycle} $\leq^p_m$ \textsc{HamPath}.

\clearpage

\appendix

\section{Long answer to part 1}
This was my original attempt in arguing both problems are in $NP$ by showing they reduce to $TSP$, and $TSP$ is in $NP$.

\subsubsection*{Argue that both problems are in $NP$.}
Consider the traveling salesman problem ($TSP$).
$TSP$ is $NP$-complete, and therefore $TSP \in NP$.
Based on the closure theorem, $NP$ is closed under $\leq^p_m$.
Therefore, given any problem $A$ that poly time many-one reduces to $TSP$ ($A \leq^p_m TSP$), we can conclude that $A \in NP$.
We claim that $\textsc{HamCycle} \leq^p_m TSP$:

Given any graph $G=(V, E)$, we define $h(G)$ as a weighted graph with weight function $W$ so that given any two vertices $x$ and $y$ in $V$, $W(x, y) = 1$ IFF $\{x , y \} \in E$, and otherwise $W(x, y) = 0$.
If $G \in \textsc{HamCycle}$, then $h(G) \in TSP$ with a path of exactly length $|V|$.
If $h(G) \in TSP$, that would mean that there exists a cycle visiting all vertices exactly once by definition, therefore $G \in \textsc{HamCycle}$.

As a result, $\textsc{HamCycle} \leq^p_m TSP$ is valid, therefore we can conclude that $\textsc{HamCycle} \in NP$.

We also claim that $\textsc{HamPath} \leq^p_m \textsc{HamCycle}$, and if this claim is true, based on it and the fact that $\textsc{HamCycle} \leq^p_m TSP$, we can claim that $\textsc{HamPath} \leq^p_m TSP$.

Given any graph $G = (V, E)$, we define $g(.)$ as a function that constructs graph $g(G) = (V^\prime , E^\prime)$, where:
\begin{align*}
    V^\prime &= V \cap \{ x \} & \\
    E^\prime &= E \cap  \{ \{ x, a \} \ | \ a \in V \} \ .
\end{align*}
If graph $G$ has a Hamilton path from some vertex $a$ to another vertex $b$, adding $x$ and edges $\{ x , a \}$ and $\{ x , b \}$ turns it into a cycle, and because $x \notin G$, the cycle is Hamilton cycle in $g(G)$.
If $g(G)$ does not have a Hamilton cycle, $G$ must also not have a Hamilton path. 
If $g(G)$ didn't have a Hamilton cycle, but $G$ had a Hamilton path, and $p$ and $q$ were two endpoints, $g(G)$ would introduce $x$, and edges between $x$ and $p$ and $x$ and $q$, therefore $g(G)$ has a Hamilton cycle, which is a contradiction.

Therefore $\textsc{HamPath} \leq^p_m \textsc{HamCycle}$, and as a result $\textsc{HamPath} \leq^p_m TSP$, and therefore $\textsc{HamPath} \in NP$.

\end{document}
